
\chapter{Introduction}\label{ch:introduction}

This lecture script was written alongside the notes/lectures given by Professor Wunderlich in the winter semester of 20/21 at Uni Leipzig.
It is not in any way a copy of the lecture, just my interpretation.
Many phenomena he covered in the lecture I have chosen to spend more detail about, while others not.

\section{Defining Quantum Technologies}\label{sec:dqt}
The class defines it as the technology that relies on the principles of quantum physics, and how properties of quantum mechanics, like quantum entanglement, quantum superposition, quantum tunneling, etc., can breed spectacular new technologies.
An example of a few discussed in the future are listed below:
\begin{myitemize}
	\item quantum computing (also classical in some cases)
	\item quantum sensors (already commercially available)
	\item quantum cryptography (decode and encode info that can not be decoded without the knowledge of the makers)
 	\item quantum simulation (drug design)
	\item quantum metrology (connected to quantum sensors)
	\item quantum imaging
\end{myitemize}

We have come a long way since computers were first being built in the 60s.
For example, the 1969 moon lander had 2800 silicon chips and 6 transistors which correspond to 70 kilobytes (kB) of core memory, whereas the Curiosity Mars rover has a fully integrated 200 MHz chip (motherboard) corresponding to 156 MB and 2 GB RAM (keep in mind these chips on Curiosity can also resist radiation of particles on the order of $2\times 10^9$eV).
The development of these integrated circuits can be accredited to Jack Kilby of Texas Instruments, who created the first integrated circuit in 1958 (winning the Nobel Prize in 2000) consisting of a single transistor.
Today a typical Intel processor is 14 nanometers in length and has 5.6 billion transistors.
The cost of a transistor has decreased to be about $\$0.3\times 10^{-9}$.

A few of the everyday technologies brought about by quantum phenomena are listed below.
\begin{myitemize}
	\item Microelectronics (USB/Flash Memory)
	\item GPS
	\item Optical Fibers
	\item Smartphones
	\item NMR/NMI (measure spin population of hydrogen in your blood)
	\item Supercomputers
	\item WI-FI
\end{myitemize}

However, as much as we are advancing here on Earth, we still have a few hindrances.
One being Moors law, i.e., the computational power of computers is doubling every few years, but the size of chips are halving.
Currently, the IC structures are about 5 nm in size, but at some point the size of these be limited, thus the need for developing new ideas.
A neat example of quantum technology today is the USB flash memory.
The idea revolves around an NOR gate, and the use of hot electron injection for writing data and quantum tunneling for the erasure of data!

In the following chapters there will be the following content described.
\begin{myitemize}
	\item Technique (devices): accelerators, detectors, ion sources
	\item Physics: interaction of ions with matter, spin physics
	\item Ion Beam techniques: implantation, analysis
	\item Application: in semiconductor physics and in the creation of quantum mechanical systems
\end{myitemize}

Mainly we will talk about ion beams, but in order to do so, we need to know a few basic principles of ions.
